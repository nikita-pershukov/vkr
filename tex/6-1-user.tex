\chapter{Инструкция для пользователя}
\label{cha:app:user}

Проект должен являться микросервисным приложением, желательно разделенным на несколько Docker образов.

Для проверки работоспособности на собственных вычислительных мощностях с помощью docker-compose, необходимо:
\begin{itemize}
    \item Установить систему контейнеризации Docker и менеджер контейнеров Docker-compose;
    \item Подготовить директорию с исходным кодом проекта;
    \item Написать docker-compose манифест (пример -- \textbf{https://docs.docker.com/compose/gettingstarted/});
    \item Запустить проект с помощью \texttt{docker-compose up};
    \item Проверить работоспособность каждого модуля проекта.
\end{itemize}

Далее необходимо подготовить проект для развертывания в кластере. Для этого необходимо:
\begin{itemize}
    \item Получить данные для авторизации в кластере проекта в GitLab;
    \item Подготовить манифесты для развертывания проекта в кластере;
    \item Настроить репозиторий с помощью полученных данных для авторизации в кластере;
    \item Запустить процесс CI/CD посредством обновления данных в репозитории.
\end{itemize}

Получить данные для авторизации в кластере проекта в GitLab можно посредством использования веб-портала АДРЕС???. На нем необходимо пройти авторизацию через единую систему авторизации НИУ ВШЭ, после чего выбрать необходимый проект и последовательно следовать инструкции на портале.

Для упрощения подготовки манифестов для развертывание проекта создан шаблон, в котором находится простое веб-приложение на языке программирования Go и манифесты для сборки и развертывания проекта. Шаблон доступен по адресу \textbf{https://git.miem.hse.ru/423/demo-projects}. В нем содержится:
\begin{itemize}
    \item Исходный код проекта -- \texttt{main.go};
    \item Dockerfile для упаковки приложения в образ -- \texttt{Dockerfile}
    \item Манифест для сборки и запуска процесса развертывания -- \texttt{.gitlab-ci.yml}
    \item Директория с манифестами для развертывания проекта в кластер -- \texttt{.helm}
\end{itemize}

Для оптимизации написания собственного Dockerfile необходимо изучить материал с примерами и лучшими практиками \textbf{https://docs.docker.com/develop/develop-images/dockerfile\_best-practices/}

Манифест для сборки и запуска процесса развертывания должен состоять из трех этапов:
\begin{itemize}
    \item \texttt{build}
    \item \texttt{push}
    \item \texttt{deploy}
\end{itemize}

На этапе \texttt{build} выполняется сборка образов Docker по инструкциям из Dockerfile. В случае, если необходимо собрать более одного образа Dockerfile, стоит расположить исходный код и Dockerfile к нему в отдельной поддиректории. И для каждого Dockerfile скорректировать инструкцию данного этапа, начинающуюся с \texttt{docker build}.

На этапе \texttt{push} выполняется отправка образов в Docker Registry. В случае, если на предыдущем этапе было собрано несколько образов, их все необходимо отправить, склонировав и скорректировав инструкцию, начинающуюся с \texttt{docker push}, несколько раз.

На этапе \texttt{deploy} выполняется развертывание образа в кластер. Он основан на содержимом директории \texttt{.helm}. В случае, когда было собрано более одного образа, необходимо склонировать и скорректировать перегрузку переменных для образов согласно содержимому файла \texttt{.helm/values.yaml}. Эти команды начинаются со слов \texttt{--set}

Далее необходимо взять за основу директорию \texttt{.helm} из шаблона и скорректировать ее содержимое под реальный проект. В файле \texttt{.helm/Chart.yaml} нужно корректно задать следующие значения:
\begin{itemize}
    \item \texttt{description}
    \item \texttt{name}
    \item \texttt{version}
\end{itemize}

После этого требует корректировки файл \texttt{.helm/values.yaml}. Его стилистка предполагает, что на каждый микросервис необходимо выделить свою подгруппу значений, как это сделано с \texttt{app}. Обязательными значениями являются:
\begin{itemize}
    \item \texttt{repository} -- можно поставить пустое значение в кавычках, так как это значение должно быть перегружено в GitLab CI/CD;
    \item \texttt{tag} -- можно поставить пустое значение в кавычках, так как это значение должно быть перегружено в GitLab CI/CD;
    \item \texttt{pullPolicy} -- предпочтительно значение \texttt{Always}, чтобы образ точно скачался при новом развертывании;
    \item \texttt{replicaCount} -- количество экземлпяров образа для запуска. Стандартное значение -- \texttt{1};
    \item \texttt{port} -- сетевой порт, на котором в образе поднят сервер.
\end{itemize}

Далее, под каждый группу значений нужно создать или скорректировать манифесты сущностей Kubernetes. Для развертывания образа необходима сущность Deployment, шаблон которой описан в \texttt{.helm/templates/deployment-app.yaml}. Для возможности взаимодействия с образом Docker по внутренней сети, необходима сущность Service -- \texttt{.helm/templates/service-app.yaml}. Для обработки трафика из внешней сети нужна сущность Ingress -- \texttt{.helm/templates/ingress.yaml}.

Последний этап подготовки проекта - настройка репозитория с помощью полученных данных для авторизации в кластере. Для этого необходимо, согласно инструкции с портала, на котором были получены данные, ввести их в настройках проекта в GitLab.
