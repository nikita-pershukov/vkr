\section{Виды архитектур инфраструктуры и приложений в ней}
\label{sec:history}

На текущий момент существует множество решений со стороны архитектуры инфраструктуры и, как следствие, приложений в ней. Для выбора конкретного решения под вышеуказанные цели необходимо рассмотреть основные из них. Далее они будут описаны по хронологии появления.

\begin{enumerate}
    \item Монолитная эра. Ей свойственны следующие аспекты:
        \begin{itemize}
            \item Приложения монолитные;
            \item Большое количество зависимостей;
            \item Долгая разработка до релиза;
            \item Все инстансы известны по именам;
                \Define{Инстанс}{экземлпяр объекта}
            \item Использование виртуализации. Это означает:
                \begin{itemize}
                    \item Один сервер -- несколько виртуальных машин (далее -- ВМ или VM)~\cite{wiki:vm};
                        \Abbrev{VM}{virtual machine (виртуальная машина)}
                        \Define{Виртуальная машина}{программная и/или аппаратная система, эмулирующая аппаратное обеспечение некоторой платформы (target — целевая, или гостевая платформа) и исполняющая программы для target-платформы на host-платформе (host — хост-платформа, платформа-хозяин) или виртуализирующая некоторую платформу и создающая на ней среды, изолирующие друг от друга программы и даже операционные системы~\cite{wiki:vm}}
                    \item Resource Management;
                        \Define{Resource management}{управление и ограничение ресурсов машины}
                    \item Изоляция окружений.
                \end{itemize}
        \end{itemize}

        Соответсвенно использовались VM:
        \begin{itemize}
            \item VMWare;
            \item Microsoft Hyper-V;
            \item VirtualBox;
            \item Qemu.
        \end{itemize}

        Подход был следующий: один большой сервер делили на несколько виртуальных машин. Это давало полную изоляцию, но недостатками были:
        \begin{itemize}
            \item Hypervisor~\cite{wiki:hypervisor};
                \Define{Гипервизор}{программа или аппаратная схема, обеспечивающая или позволяющая одновременное, параллельное выполнение нескольких операционных систем на одном и том же хост-компьютере~\cite{wiki:hypervisor}}
            \item Большие образы;
            \item Как следствие больших образов с разным ПО -- медленное управление VM.
        \end{itemize}
    \item На смену им пришла виртуализация на уровне ядра с помощью следующих инструментов:
        \begin{itemize}
            \item OpenVZ;
            \item Systemd-nspawn;
            \item LXC.
        \end{itemize}

        Но остались прежние проблемы:
        \begin{itemize}
            \item Большие образы с операционной системой (далее -- ОС или OS) с большим количеством ПО;
                \Abbrev{OS или ОС}{операционная система}
                \Abbrev{ПО}{программное обеспечение}
            \item Нет стандарта упаковки и доставки;
            \item DLL Hell~\cite{dick2018dll}.
                \Define{DLL Hell}{сбой, возникающий, когда одна часть программного обеспечения ведет себя не так, как ожидалось второй частью программного обеспечения, что в некотором роде <<зависит>> от действия первого~\cite{dick2018dll}}
        \end{itemize}
    \item Но далее пришли контейнеры. Разница между VM и контейнером:
        \begin{itemize}
            \item Виртуальная машина подразумевает виртуализацию железа для запуска гостевой ОС;
            \item Контейнер использует ядро хостовой ОС;
            \item В VM может работать любая ОС;
            \item В контейнере может работать только GNU/Linux (с недавних пор и Windows);
            \item VM хороша для изоляции;
            \item Контейнеры не подходят для изоляции.
        \end{itemize}

        В итоге мы приходим к ситуации, изображенной далее \imref{container-vs-vm}:
        \includeimg{container-vs-vm}{width=1\linewidth}{Сравнение VM и контейнеров.}

        Способ реализации контейнеризации:
        \begin{itemize}
            \item Namespaces:
                \Define{Namespaces}{пространства имен}
                \begin{itemize}
                    \item Process ID (далее -- PID);
                        \Abbrev{PID}{Process ID}
                        \Define{Process ID}{идентификатор процесса}
                    \item Networking;
                        \Define{Networking или network}{сеть}
                    \item Mount;
                        \Define{Mount}{монтирование, например дисков как директорий. Способ подключения различных хранителей данных в OS GNU/Linux}
                    \item User.
                        \Define{User}{пользователь}
                \end{itemize}
            \item Control Groups (далее -- Cgroup):
                \Abbrev{Cgroup}{control group}
                \Define{Control group}{управление (контроль) групп}
                \begin{itemize}
                    \item Memory;
                        \Define{Memory}{память}
                    \item CPU;
                        \Abbrev{CPU}{Central Processing Unit}
                        \Define{Central Processing Unit}{процессор}
                    \item Block Input/Output (далее -- I/O);
                        \Abbrev{I/O}{input and output}
                        \Define{Block Input/Output}{обращение на считывание и запись с хранителей данных}
                    \item Network.
                \end{itemize}
        \end{itemize}

        В итоге получается следующая логика:
        \begin{itemize}
            \item Один процесс -- один контейнер;
            \item Все зависимости в контейнере;
            \item Чем меньше образ контейнера (файл, включающий зависимости, сведения, конфигурацию для дальнейшего развертывания и инициализации контейнера)~\cite{def:docker} -- тем лучше;
                \Define{Image или образ контейнера}{файл, включающий зависимости, сведения, конфигурацию для дальнейшего развертывания и инициализации контейнера~\cite{def:docker}}
            \item Инстансы становятся эфемерными.
        \end{itemize}

        И в 2014-2015 годах Docker~\cite{демидова2019использование} приобрел популярность за счет следующих аспектов:
        \Define{Docker}{программное обеспечение для автоматизации развертывания и управления приложениями в среде виртуализации на уровне операционной системы~\cite{демидова2019использование}}
        \begin{itemize}
            \item Меняет философию подхода к разработке;
            \item Стандартизует упаковку приложения;
            \item Решает вопрос зависимостей;
            \item Гарантирует воспроизводимость;
            \item Обеспечивает минимум дополнительных средств для использования.
        \end{itemize}

        !!! МБ ДОБАВИТЬ ПРО CONTAINERD

\end{enumerate}

На основе этих данных можно сделать вывод, что использование контейнеризации и, в частности, Docker позволит сущетсвенно сократить метрики по прохождению этапов от разработки кода до промышленного использования при использовании микросервисной архитектуры.
\Define{Микросервисная архитектура}{blablabla}
