\section{Разработка архитектуры и выбор инструментария}
\label{sec:tools}

На основе всего вышеописанного можно сделать вывод, что на текущий момент среди архитектур программных проектов лидирует микросервисная архитектура, за исключением ситуаций, когда она неприменима.

Среди систем контроля версий лидером является Git. А самой востребованной реализацией сервера Git, которую можно расположить на собственных вычислительных мощностях, является GitLab.

В микросервисной архитектуре для удобства упаковки, доставки и развертывания продукта используются контейнеры. Лидер среди них -- Docker.

Для того чтобы расположить контейнеры Docker на нескольких серверах, удобно их администрировать, масштабировать, получать с них метрики для мониторинга, а также других связанных задач, используется оркетстратор контейнеров. На основе материалов, рассмотренных в разделе \ref{sec:lit-rev}, предпочтение отдано Kubernetes.

Для удобства работы с Kubernetes был выбраны следующие вспомогательные компоненты:
\begin{itemize}
    \item NFS в роли хранилища данных;
    \item Traefik в качестве реверс-прокси;
    \item GitLab Runner как инструмента CI/CD вместе с Helm;
        \Define{GitLab Runner}{blablabla}
    \item CertManager в роли средства атоматизации получения сертификатов.
        \Define{CertManager}{blablabla}
\end{itemize}

Ниже приведен пример простого приложения в микросервисной архитектуре \imref{simple-microservice}: пользователь взаимойдествует с фронтенд частью приложения, которая общается с бэкендом, а тот, в свою очередь, обменивается данными с базой данных.
\includeimg{simple-microservice}{width=1\linewidth}{Простое микросервисное приложение.}

Далее изображена схема как трафик проходит от запроса пользователя до первого Pod, который будет обрабатывать трафик \imref{4m-user-2-pod}: в начале трафик попадает на Ingress Controller, который является реверс-прокси.
\Define{Реверс-прокси}{blablabla}
Тот, в соответствии с правилами Ingress, перенаправляет их на Endpoints, которые созданы с помощью Service. А Endpoints являются списком IP адресов Pod, на которые нужно перенаправить трафик для его обработки.
\includeimg{4m-user-2-pod}{width=1\linewidth}{Передача трафика от пользователя внутри Kubernetes.}

Если совместить пример простого микросервисного приложения \imref{simple-microservice} и схему прохода трафика внутри Kubernetes, то получается более реалистичная иллюстрация происходящего \imref{4m-user-2-3-step}:
\includeimg{4m-user-2-3-step}{width=1\linewidth}{Передача трафика от пользователя внутри Kubernetes для микросервисного приложения.}

При первом приближении так выглядит взаимодействие разработчиков с новой архитектурой \imref{dev-2-k8s}: разработчики взаимодействуют большую часть времени с системой контроля версий GitLab. Далее, в рамках CI/CD, собирается артефакт, который разворачивается в кластер Kubernetes с помощью Helm.
\includeimg{dev-2-k8s}{width=1\linewidth}{Упрощенная схема взаимодействия разработчиков с архитектурой.}

!!! Картинки с разными уровнями погружений и т.д. в новую архитектуру с пояснениями
