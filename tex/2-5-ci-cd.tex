\section{CI/CD}
\label{sec:ci-cd}

После этого была начата разработка тестового проекта для развертывания в кластере с помощью CI/CD. За основу был взят минимальный веб-сервер на языке программирования Go -- язык программирования, начало которого было положено в 2007 году сотрудниками компании Google~\cite{def:go},
\Define{Go}{язык программирования, начало которого было положено в 2007 году сотрудниками компании Google~\cite{def:go}}
который отдавал статичную страницу. Далее был написан Dockerfile -- описание правил по сборке образа, в котором первая строка указывает на базовый образ~\cite{def:docker},
\Define{Dockerfile}{описание правил по сборке образа, в котором первая строка указывает на базовый образ~\cite{def:docker}}
с командами для сборки и запуска веб-сервера. Первая его часть основан на Docker Image
\texttt{golang:1.8-alpine}. Далее код копируется внутрь Docker Image посредством директивы texttt{ADD} и компилируется с помощью директивы \texttt{RUN}. После этого берется второй Docker Image \texttt{alpine:latest}, в который переносится скомпилированный веб-сервер директивой \texttt{COPY}. Также объявляется порт для прослушивания трафика директивой \texttt{ENV}. И последней директивой \texttt{CMD} запускается веб-сервер.

Далее было необходимо написать Helm-chart для развертывания контейнера Docker и всех необходимых для его работы сущностей Kubernetes в кластере. В качестве ключевых файлов выступают \texttt{.helm/values.yaml} и \texttt{.helm/Chart.yaml}, в которых содержатся:

\begin{itemize}
    \item Название проекта.
    \item Описание проекта.
    \item Версия проекта.
    \item Версия используемого API~\cite{def:api}.
        \Abbrev{API}{Application Programming Interface}
        \Define{Application Programming Interface}{описание способов (набор классов, процедур, функций, структур или констант), которыми одна компьютерная программа может взаимодействовать с другой программой~\cite{def:api}}
    \item Название Docker Image.
    \item Тег Docker Image.
    \item Политика скачивания образа.
    \item Количество копий образа, которое должно быть запущено.
    \item Порт, на котором веб-сервер образа слушает трафик.
\end{itemize}

На основе этих переменных все остальные файлы в директории \texttt{.helm/templates} будут описывать сущности кластера Kubernetes:

\begin{itemize}
    \item \texttt{helm/templates/deployment-app.yaml}
    \item \texttt{helm/templates/service-app.yaml}
    \item \texttt{helm/templates/ingress.yaml}
\end{itemize}

Далее был написан файл \texttt{.gitlab-ci.yml}, который является YAML-манифестом для описания этапов CI/CD для развертывание проекта в кластере. Он состоит из трех этапов:

\begin{itemize}
    \item \texttt{build}.
    \item \texttt{push}.
    \item \texttt{deploy}.
\end{itemize}

Этап \texttt{build} выполняется в одну команду \texttt{docker build} с указанием тега самим GitLab Runner.

Этап \texttt{push} требует предварительной аутентификации в Docker Registry -- зарезервированный сервер, используемый для хранения docker-образов~\cite{def:docker},
\Define{Docker Registry}{зарезервированный сервер, используемый для хранения docker-образов~\cite{def:docker}}
для этого был взят пример из официальной документации по GitLab. После этого выполняется \texttt{docker push}.

Этап \texttt{deploy}, относящийся к CD уже намного сложнее. Для этого требуется задать в GitLab API
URL
\Abbrev{URL}{blablabla}
\Define{Uniform Resource Locator}{blablabla}
кластера Kubernetes, а также его токен.
\Define{Токен}{blablabla}
После чего в начале выполнения этапа мы задаем эти переменные в окружение с помощью \texttt{kubectl},
\Define{kubectl}{blablabla}
после чего происходит проверка на то, существуюет ли уже данный проект в кластере. Если проект существет, то будет выполнена команда \texttt{helm upgrade}, в противном случае будет выполнена команда \texttt{helm install}. Детальнее эти вещи можно увидеть в ПРИЛОЖЕНИЕ ТАКОМ-ТО.
