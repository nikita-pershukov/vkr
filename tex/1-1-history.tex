\section{Общие сведения о архитектуре инфраструктуры и приложений в ней}
\label{sec:history}

На текущий момент существует множество решений со стороны архитектуры инфраструктуры и, как следствие, приложений в ней. Для выбора конкретного решения под вышеуказанные \textbf{Указать цели выше} цели необходимо рассмотреть основные из них. Далее они будут описаны по хронологии появления.

\begin{enumerate}
    \item Монолитная эра. Ей свойственны следующие аспекты:
        \begin{itemize}
            \item Приложения монолитные.
            \item Куча зависимостей.
            \item Долгая разработка до релиза.
            \item Все инстансы известны по именам.
                \Define{Инстанс}{экземлпяр объекта}
            \item Использование виртуализации. Это означает:
                \begin{itemize}
                    \item Один сервер -- несколько VM \cite{wiki:vm}.
                        \Abbrev{VM}{virtual machine (виртуальная машина)}
                        \Define{Виртуальная машина}{программная и/или аппаратная система, эмулирующая аппаратное обеспечение некоторой платформы (target — целевая, или гостевая платформа) и исполняющая программы для target-платформы на host-платформе (host — хост-платформа, платформа-хозяин) или виртуализирующая некоторую платформу и создающая на ней среды, изолирующие друг от друга программы и даже операционные системы \cite{wiki:vm}}
                    \item Resource Management.
                        \Define{Resource management}{управление и ограничение ресурсов машины}
                    \item Изоляция окружений.
                \end{itemize}
        \end{itemize}
        Соответсвенно использовались VM:
        \begin{itemize}
            \item VMWare.
            \item Microsoft Hyper-V.
            \item VirtualBox.
            \item Qemu.
        \end{itemize}
        Подход был следующий: один большой сервер делили на несколько виртуальных машин. Это давало полную изоляцию, но недостатками были:
        \begin{itemize}
            \item Hypervisor \cite{wiki:hypervisor}.
                \Define{Гипервизор}{программа или аппаратная схема, обеспечивающая или позволяющая одновременное, параллельное выполнение нескольких операционных систем на одном и том же хост-компьютере \cite{wiki:hypervisor}}
            \item Большие образы.
            \item Как следствие больших образов с разным ПО -- медленное управление VM.
        \end{itemize}
    \item На смену им пришла виртуализация на уровне ядра с помощью следующих инструментов:
        \begin{itemize}
            \item OpenVZ.
            \item Systemd-nspawn.
            \item LXC.
        \end{itemize}
        Но остались прежние проблемы:
        \begin{itemize}
            \item Большие образы с OS с большим количеством ПО.
                \Abbrev{OS или ОС}{операционная система}
                \Abbrev{ПО}{программное обеспечение}
            \item Нет стандарта упаковки и доставки.
            \item DLL Hell \cite{dick2018dll}.
                \Define{DLL Hell}{сбой, возникающий, когда одна часть программного обеспечения ведет себя не так, как ожидалось второй частью программного обеспечения, что в некотором роде "зависит" от действия первого \cite{dick2018dll}}
        \end{itemize}
    \item Но далее пришли контейнеры. Разница между VM и контейнером:
        \begin{itemize}
            \item Виртуальная машина подразумевает виртуализацию железа для запуска гостевой ОС.
            \item Контейнер использует ядро хостовой ОС.
            \item В VM может работать любая ОС.
            \item В контейнере может работать только GNU/Linux (с недавних пор и Windows).
            \item VM хороша для изоляции.
            \item Контейнеры не подходят для изоляции.
        \end{itemize}
        В итоге мы приходим к ситуации, изображенной на \imref{container-vs-vm}:
        \includeimg{container-vs-vm}{width=1\linewidth}{Сравнение VM и контейнеров}
        \newpage
        Способ реализации контейнеризации:
        \begin{itemize}
            \item Namespaces:
                \Define{Namespaces}{пространства имен}
                \begin{itemize}
                    \item PID.
                        \Abbrev{PID}{Proccess ID}
                        \Define{Process ID}{идентификатор процесса}
                    \item Networking.
                        \Define{Networking или network}{сеть}
                    \item Mount.
                        \Define{Mount}{монтирование, например дисков как директорий. Способ подключения различных хранителей данных в OS GNU/Linux}
                    \item User;
                        \Define{User}{пользователь}
                \end{itemize}
            \item Control Groups:
                \Abbrev{Cgroup}{control group}
                \Define{Control group}{управление (контроль) групп}
                \begin{itemize}
                    \item Memory.
                        \Define{Memory}{память}.
                    \item CPU
                        \Abbrev{CPU}{central processing unit}
                        \Define{Central processing unit}{процессор}
                    \item Block I/O;
                        \Abbrev{I/O}{input and output}
                        \Define{Block input and output}{обращение на считывание и запись с хранителей данных}
                    \item Network;
                \end{itemize}
        \end{itemize}
        В итоге получается следующая логика:
        \begin{itemize}
            \item Один процесс -- один контейнер.
            \item Все зависимости в контейнере.
            \item Чем меньше образ контейнера -- тем лучше.
                \Define{Image или образ контейнера}{архив файловой системы, который используется для запуска контейнера из него}
            \item Инстансы становятся эфемерными.
        \end{itemize}
        И в 2014-2015 годах пришел расцвет Docker \cite{демидова2019использование}, который:
        \Define{Docker}{программное обеспечение для автоматизации развертывания и управления приложениями в среде виртуализации на уровне операционной системы \cite{демидова2019использование}}
        \begin{itemize}
            \item Меняет философию подхода к разработке.
            \item Стандартизует упаковку приложения.
            \item Решает вопрос зависимостей.
            \item Гарантирует воспроизводимость.
            \item Обеспечивает минимум дополнительных средств для использования.
        \end{itemize}
\end{enumerate}

По своей сути Docker -- это обертка или надстройка над Namespaces и Cgroup, которые существуют как родные инстументы изоляции в OS GNU/Linux, которые упрощают запуск процесса в изолированном пространстве.

\clearpage
\textbf{!!! Переписать}

Выбранный стек технологий является одним из основных на текущий момент для разработки и развертывания приложений во всем мире. Но отдельного упоминания заслуживают причины, которые привели к этой ситуации:

\subsubsection{Docker}

\textbf{Docker} - это программное обеспечения для автоматизации развертывания и управления приложениями в средах с поддрежкой контейнеризации.\textit{Но что это значит?} Для понимания этого инструмента необходимо посмотреть, что было до него:
\begin{itemize}
    \item Монолитная эра:
        \begin{itemize}
            \item Приложения монолитные;
            \item Куча зависимостей;
            \item Долгая разработка до релиза;
            \item Все инстансы известны по именам;
            \item Использование виртуализации:
                \begin{itemize}
                    \item Один сервер - несколько \textit{VM};
                    \item \textit{Resource Management};
                    \item Изоляция окружений.
                \end{itemize}
        \end{itemize}
        Соответсвенно использовались \textit{VM}:
        \begin{itemize}
            \item VMWare;
            \item Microsoft Hyper-V;
            \item VirtualBox;
            \item Qemu.
        \end{itemize}
        Подход был такой: один большой сервер делили на несколько виртуальных машин. Это давало полную изоляцию, но недостатками были:
        \begin{itemize}
            \item Hypervisor;
            \item Большие образы;
            \item И как следствие больших образов с разным ПО - медленное управление \textit{VM}.
        \end{itemize}
    \item На смену им пришла виртуализация на уровне ядра:
        \begin{itemize}
            \item OpenVZ;
            \item Systemd-nspawn;
            \item LXC.
        \end{itemize}
        Но остались прежние проблемы:
        \begin{itemize}
            \item Большие образы с \textit{OS} с большим количеством ПО.
            \item Нет стандарта упаковки и доставки.
            \item \textit{DLL Hell}.
        \end{itemize}
    \item Но далее пришли контейнеры. Разница между \textit{VM} и контейнером:
        \begin{itemize}
            \item Виртуальная машина подразумевает виртуализацию железа для запуска гостевой ОС.
            \item Контейнер использует ядро хостовой ОС.
            \item В \textit{VM} может работать любая ОС.
            \item в контейнере может работать только GNU/Linux (с недавних пор и Windows).
            \item \textit{VM} хороша для изоляции.
            \item Контейнеры не подходят для изоляции.
        \end{itemize}
        %В итоге мы приходим к ситуации, изображенной на \imref{container-vs-vm.png}:
        %\includeimg{container-vs-vm.png}{width=1\linewidth}{Сравнение \textit{VM} и контейнеров}
        \newpage
        Способ реализации контейнеризации:
        \begin{itemize}
            \item \textit{Namespaces}:
                \begin{itemize}
                    \item \textit{PID};
                    \item \textit{Networking};
                    \item \textit{Mount};
                    \item \textit{User};
                    \item etc.
                \end{itemize}
            \item Control Groups:
                \begin{itemize}
                    \item Memory;
                    \item CPU;
                    \item Block \textit{I/O};
                    \item \textit{Network};
                    \item etc.
                \end{itemize}
        \end{itemize}
        В итоге получается следующая логика:
        \begin{itemize}
            \item Один процесс - один контейнер.
            \item Все зависимости в контейнере.
            \item Чем меньше образ контейнера - тем лучше.
            \item Инстансы становятся эфемерными.
        \end{itemize}
        И в 2014-2015 годах пришел расцвет \textit{Docker}, который:
        \begin{itemize}
            \item Меняет философию подхода к разработке.
            \item Стандартизует упаковку приложения.
            \item Решает вопрос зависимостей.
            \item Гарантирует воспроизводимость.
            \item Обеспечивает минимум дополнительных средств для использования.
        \end{itemize}
\end{itemize}
По своей сути \textit{Docker} - это обертка или надстройка над \textit{Namespaces} и \textit{Cgroup}, которые существуют как родные инстументы изоляции в \textit{OS} GNU/Linux, которые упрощают запуск процесса в изолированном пространстве.

\subsubsection{Kubernetes}

\textbf{Kubernetes} - оркестратор контейнеров. \textit{Зачем он нужен?} У нас уже есть \textit{Docker}, который удобен для разработки и развертки приложений, особенно в формате микросервисов. Его можно сравнить с фурой, на которую грузят один контейнер и которая его везет. Но этого зачастую недостаточно. Есть \textit{Docker} Compose, который может запускать несколько контейнеров, взаимодействующих друг с другом и внешним миром. Его можно сравнить с паровозом, который может везти несколько контейнеров. В свое время \textit{Kubernetes} правильно сравнивать с морским портом, где находится огромное количество контейнеров, которые распределены по разным складам, перемещаются между складами при необходимости, чинит их при поломке. То есть, вне зависимости от масштаба и количества контейнеров, \textit{Kubernetes} позволяет их гибко разварачивать, управлять, перемещать и масштабировать.
Его основные преимущества:
\begin{itemize}
    \item \textit{Immutable} - неизменямая структура. Эта идея уже была реализована в \textit{Docker} касательно \textit{Docker} Image - он не изменяется в процессе работы. В \textit{Kubernetes} же неизменна вся структура.
    \item \textit{Self-healing} - каждый компонент отвечает за свою часть инфрастуктуры и постоянно поддерживает ее в актуальном состоянии.Например, при падении одного из нескольких серверов те контейнеры, что были запущены на упавшем сервере, будут запущены на одном из оставшимся в живых сервере.
    \item \textit{Declarative} - описание не того, что надо сделать (императивный подход), а как должен выглядеть итоговый результать. В случае \textit{Kubernetes} все управление происходит посредством \textit{Yaml} манифестов.
    \item Каждый компонент инфраструктуры независим от других и полагается на их \textit{SLA}.
\end{itemize}
По своей сути \textit{Kubernetes} - это надстройка над \textit{Docker}, которая позволяет управлять огромным количеством контейнеров, распределенных на разных серверах.

\subsubsection{Helm}

\textbf{Helm} - инструмент, который помогает управлять приложениями в \textit{Kubernetes}. \textit{Но зачем нам еще помощь, когда уже есть Docker и Kubernetes, которые сильно упрощают жизнь?} Для этого надо посмотреть на логику работы с приложением. Допустим есть приложение, которое состоит из нескольких \textit{YAML} манифестов для \textit{Kubernetes}, которые уже настроены, и их можно применить с помощью \textit{kubectl apply -f path/to/dir}. Вроде все хорошо. Но ведь приложение растет, оно может начать занимать сотни \textit{YAML} манифестов, а помнить все переменные окружения и прочие вещи станет достаточно сложно. Самый простой пример: замена \textit{labels} и \textit{selector} во всех манифеста одного микросервиса. Есть варианты использовать \textit{sed} или \textit{ansible}, и они решат проблему запуска чего-то нового. Но что делать, если что-то пошло не так? То есть как откатить на предыдущую версию или как контроллировать процесс релиза? И тут на помощь приходит \textit{Helm}. Он при применении создает артефакт с указанной версии и сохраняет его где-то. И в случае, если что-то пошло не так, возможно взять сохраненный предыдущий артефакт и запустить из него стабильную версию. При этом создатели \textit{Helm} позиционируют его как "пакетный менеджер", с помощью которого можно скачать, установить, удалить приложение вместе с его зависимостями. Он, как и сам \textit{Kubernetes}, декларативный. Также, он умеет не просто применить \textit{YAML} манифесты, но и отследить процесс запуска всего необходимого, а в случае ошибки, откатиться назад автоматически.
