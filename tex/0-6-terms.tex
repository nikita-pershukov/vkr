\chapter{СТАРАЯ Терминология}
\label{cha:terms}

% \textbf{Инстанс} - экземлпяр объекта, в данном случае одного сервера.
% 
% \textbf{VM} или \textbf{ВМ} - виртуальная машина.
% 
% \textbf{Resource Management} - управление и/или ограничение ресурсов машины.
% 
% \textbf{OS} или \textbf{ОС} - операционная система.
% 
% \textbf{Namespace} - пространство имен.
% 
% \textbf{Cgroup (Control Group)} - управление (контроль) групп.
% 
% \textbf{PID (Process ID)} - идентификатор процесса.
% 
% \textbf{Network} - сеть.
% 
% \textbf{Mount} - монтирование, например дисков как директорий. Способ подключения различных хранителей данных в \textit{OS} GNU/Linux.
% 
% \textbf{User} - пользователь.
% 
% \textbf{I/0 (Input/Output)} - в данном случае обращение на считывание и запись с хранителей данных.
% 
% \textbf{DLL Hell} - проблема конфликта зависимостей в виде одной и той же библиотеки, но разных версий.
%
%\textbf{Image} или \textbf{Образ контейнера} - "архив" файловой системы, который используется для запуска контейнера из него.
%
\textbf{Immutable} - неизменный.

\textbf{Self-healing} - самолечение.

\textbf{Declarative} - декларативность.

\textbf{SLA} - доступность.

\textbf{Daemon} или \textbf{Демон} - процесс, запущенный в фоне.

\textbf{CLI (Command line interface)} - консольный интерфейс приложения.

\textbf{FS (File system)} - файловая система раздела.

\textbf{Hash} или \textbf{Хэш} - результат выполнения некоторой математической хэш-функции.

\textbf{Tag} или \textbf{Тэг} - некоторая отметка на чем-либо для более удобного поиска объектов.

\textbf{Yaml (YAML Ain't Markup Language)} -"дружественный" формат сериализации данных, концептуально близкий к языкам разметки.

\textbf{SSD (Solid State Drive)} или \textit{Твердотельный диск} - вид накопителя, хранящего данные.

\textbf{REST API} - архитектурный стиль взаимодействия компонентов распределенного приложения в сети.

\textbf{Garbage Collector} - "сборщик мусора".

\textbf{QoS (Quality of Service)} - органичения на что-либо, разграничающие по ролям.

\textbf{Requested resources} - запрошенные ресурсы.

\textbf{Netwrok Policies} - сетевые политики.

\textbf{Iptables} - обертка над netfilter.

\textbf{Ipvs} - обертка над netfilter.

\textbf{OSI} - сетевая модель стека сетевых протоколов.

\textbf{VRRP (Virtual Router Redundancy Protocol)} - сетевой протокол, предназначенный для увеличения доступности маршрутизаторов.

\textbf{SSL/TLS (Secure Sockets Layer) (Transport Layer Security)} - криптографические протоколы, обеспечивающие защищенную передачу данных между узлами в сети Интернет.

\textbf{Hook} - перехват чего-либо.

\textbf{Git} - одна из систем контроля версий.

\textbf{HTTP (HyperText Transfer Protocol)} - протокол прикладного уровня передачи данных.

\textbf{Tar} - архиватор.

\textbf{GET Requests} - метод \textit{HTTP}.
