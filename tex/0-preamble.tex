\sloppy

\tolerance=1
\emergencystretch=\maxdimen
\hyphenpenalty=10000
\hbadness=10000

% Настройки стиля ГОСТ 7-32
% Для начала определяем, хотим мы или нет, чтобы рисунки и таблицы нумеровались в пределах раздела, или нам нужна сквозная нумерация.
%\EqInChapter % формулы будут нумероваться в пределах раздела
%\TableInChapter % таблицы будут нумероваться в пределах раздела
%\PicInChapter % рисунки будут нумероваться в пределах раздела

\usepackage[tableposition=top]{caption}
\usepackage{subcaption}
\DeclareCaptionLabelFormat{gostfigure}{Рис. #2}
\DeclareCaptionLabelFormat{gosttable}{Таблица #2}
\DeclareCaptionLabelFormat{gostlisting}{Листинг #2}
\DeclareCaptionLabelSeparator{gost}{.~}
\captionsetup{labelsep=gost}
\captionsetup[figure]{labelformat=gostfigure}
\captionsetup[table]{labelformat=gosttable}
\captionsetup[listing]{labelformat=gostlisting}
\renewcommand{\thesubfigure}{\asbuk{subfigure}}

% Добавляем гипертекстовое оглавление в PDF
\usepackage[
bookmarks=true, colorlinks=true, unicode=true,
urlcolor=black,linkcolor=black, anchorcolor=black,
citecolor=black, menucolor=black, filecolor=black,
]{hyperref}

% Изменение начертания шрифта --- после чего выглядит таймсоподобно.
% apt-get install scalable-cyrfonts-tex

\IfFileExists{cyrtimes.sty}
    {
        \usepackage{cyrtimespatched}
    }
    {
        % Если нет Times
    }

\usepackage{graphicx} % Пакет для включения рисунков
\DeclareGraphicsExtensions{.jpg,.pdf,.png}
\usepackage{geometry}
%\RequirePackage[left=20mm,right=10mm,top=20mm,bottom=20mm,headsep=0pt]{geometry}
\geometry{left=20mm}
\geometry{right=10mm}
\geometry{top=20mm}
\geometry{bottom=20mm}

\usepackage{afterpage}

\usepackage{url}
\usepackage{tabularx}

% Произвольная нумерация списков.
\usepackage{enumerate}

% Мультиколонки
\usepackage{multicol}

% Исходный код
\usepackage[all]{hypcap}
\usepackage{listings}
\usepackage[newfloat]{minted}

\setcounter{tocdepth}{2} %Подробность оглавления
%4 это chapter, section, subsection, subsubsection и paragraph
%3 это chapter, section, subsection и subsubsection
%2 это chapter, section, и subsection
%1 это chapter и section
%0 это chapter.

\usepackage{totcount}
\regtotcounter{chapter}
\regtotcounter{table}
\regtotcounter{listing}
\regtotcounter{figure}

\newtotcounter{citenum}

\def\oldbibitem{} \let\oldbibitem=\bibitem
\def\bibitem{\stepcounter{citenum}\oldbibitem}

\usepackage{lastpage}

%\makeatletter
%\renewcommand{\@seccntformat}[1]{\csname the#1\endcsname \quad}
%\makeatother

% Include image
\newcommand{\includeimg}[3]{
    \begin{figure}[H]
        \centering
        \includegraphics[#2]{../graphics/img/#1}
        \caption{#3}
        \label{#1}
    \end{figure}
}

\newcommand{\imref}[1]{(рис.~\ref{#1})}

\definecolor{codegreen}{rgb}{0,0.6,0}
\definecolor{codegray}{rgb}{0.5,0.5,0.5}
\definecolor{codepurple}{rgb}{0.58,0,0.82}

\lstdefinestyle{code_basic}{
    commentstyle=\color{codegreen},
    keywordstyle=\color{magenta},
    numberstyle=\tiny\color{codegray},
    stringstyle=\color{codepurple},
    basicstyle=\small,
    basicstyle=\scriptsize,
    basicstyle=\normalsize,
    breakatwhitespace=false,
    breaklines=true,
    keepspaces=true,
    numbers=left,
    numbersep=5pt,
    showspaces=false,
    showstringspaces=true,
    showtabs=false,
    tabsize=2
}

\lstset{style=code_basic}
\lstset{extendedchars=\true}
\lstset{inputencoding=utf8}
