\section{Traefik}
\label{sec:traefik}

Далее необходимо было настроить Ingress Controller для промышленного использования, то есть обработки сетевого трафика. В качестве этого инструмента был выбран Traefik.
\Define{Traefik}{blablabla}
Он, в сравнении с предлагаемым по умолчанию nginx, является намного более современным, лишенным некоторых ощутимых недостатков, которые присущи nginx. Его настройка также заслуживает внимания:

Файл values.yaml (ПРИЛОЖЕНИЕ ТАКОЕ-ТО):
\begin{itemize}
    \item \texttt{ingressClass[enabled]: true}

        Этот параметр означает, что будет создана сущность IngressClass в кластере Kubernetes для того, чтобы он корректно увидел новый Ingress Controller.

    \item \texttt{ingressClass[isDefaultClass]: true}

        Задание Traefik как Ingress Controller по умолчанию.

    \item \texttt{pilot[enabled]: true}

        Эта опция включает возможность управления и мониторинга активности Traefik посредством веб-интерфейса. За ней следует еще одна, содержащая токен.

    \item \texttt{volumes}

        Этот раздел параметров указывает сущности Kubernetes, которые необходимо подключить к Pod Traefik. На данном этапе это используется, чтобы подключить wildcard сертификаты
        \Define{Wildcard сертификат}{blablabla}
        для сайтов.

    \item \texttt{ports[web][hostPort]: 80}

        Это значение позволяет включить проброс сетевого трафика с 80-го порта инстанса, на котором будет находиться Pod Traefik на заданный по умолчанию порт самого Pod Traefik.

    \item \texttt{ports[web][redirectTo]: websecure}

        Эта опция включает перенаправления трафика с HTTP на HTTPS.

    \item \texttt{ports[websecure][hostPort]: 443}

        Аналогично подобной опции в разделе \texttt{web}, позволяет проброс трафика с хостовой машины внутрь Pod Traefik.

    \item \texttt{ports[websecure][tls][enabled]: true}

        Эту опцию необходимо включать для того, чтобы работали сертификаты для HTTPS. Далее, в разделе tls,
        \Define{TLS}{blablabla}
        указываются пути до сертификатов и обслуживаемые доменные имена.

    \item \texttt{service[enabled]: false}

        Так как у нас Traefik является входным узлом, то для него не требуется сущность Service
        \Define{Service}{blablabla}
        Kubernetes.

    \item \texttt{persistence[enabled]: true}

        Использование PerstistentVolumeClaim Kubernetes для хранения TLS сертификатов. Далее также необходимо указать корректный StorageClass, который был создан ранее.

    \item \texttt{podSecurityPolicy[enabled]: true}

        Включение аспектов безопасности Pod Traefik.

    \item \texttt{resources}

        Этот раздел аргументов позволяет задать минимальные требования по ресурсам для запуска Traefik, а также его лимиты, за которые он не сможет выйти по потреблению.

    \item \texttt{nodeSelector}

        С помощью сущности NodeSelector
        \Define{NodeSelector}{blablabla}
        можно указать, на каком хосте или группе хостов должен оказаться конкретный Pod. В данном случае, это позволяет расположить Pod Traefik именно там, куда проброшен трафик из Интернета.

    \item \texttt{tolerations}

        С помощью данной сущности Kubernetes можно обойти какие-либо ограничения. В текущем случае, это ограничение на планирование запуска Pod на master ноде.

\end{itemize}
