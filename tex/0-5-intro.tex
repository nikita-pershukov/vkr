\Introduction

В настоящее время существует спрос на реализацию решений по автоматизации процессов разработки и развертывания программных продуктов. Для этого используются такие средства, как:

\begin{itemize}
    \item Система контроля версий.
        \Define{Система контроля версий}{blablabla}
    \item Средства непрерывной интеграции и доставки артефактов.
        \Define{Средства непрерывной интеграции и доставки артефактов}{blablabla}
    \item Система контенеризации.
    \item Оркестрация контейнеров.
    \item Вспомогательные инструменты.
\end{itemize}

На текущий момент в МИЭМ активно развивается проектная деятельность. Но изначально инфраструктура не была готова к такому большому количеству проектов и команд, реализующих эти проекты. В первую очередь, необходима была платформа для удобного хранения разработок, то есть система контроля версий. После этого данные наработки необходимо где-то развернуть. Ранее это представляло большую многоэтапную задачу, от поиска локальных администраторов и получения мощностей, до получение доменного имени для публикации проекта для промышленного использования.

Цель работы -- оптимизация командного взаимодействия и уменьшение метрик времени вывода в продуктив при разработке программных продуктов посредством переиспользования ресурсов контейнерных сред и общих инфраструктурных кластеров.
